% %%%%%%%%%%%%%%%%%%%%%%%%%%%%%%%%%%%%
% A LaTeX template for the technical essay of TTM4137 Wireless Security
% Stig F. Mjolsnes, 01.09.2012
%%%%%%%%%%%%%%%%%%%%%%%%%%%%%%%%%%%%%
\documentclass[a4paper,11pt]{article}

\usepackage[plain]{fullpage}
\usepackage{graphicx}  %This enables the inclusion of pdf graphic files in figures
\usepackage[hidelinks]{hyperref} % Make links in your document click-able NB: must be loaded before the caption-package
\usepackage{caption}
\usepackage{subcaption}
\usepackage{wrapfig} 


\title{DNS Tunneling}
\author{Terje Kristoffer H. Skow \\
	\texttt{terjeks@stud.ntnu.no}\\
	TTM4137 Wireless Security Technical Essay}
\date{\today}

\begin{document}
\maketitle

\section{Introduction}


Domain Name System (DNS) is one of the backbone systems of the internet. It is a protocol that is used to lookup a domain name's IP-address, which the network needs to route http traffic. In later years it has been discovered that it has weaknesses that is easy to exploit. The exploit which is going to be looked at in essay is DNS tunneling. DNS tunneling started with people who wanted internet access at hotels and cafés without paying for it. It is now used in "Command and Control" attack where the commands can be sendt encrypted over DNS masked as a regular response, and to transfer data undetected out of a secure network. With the use of smartphones using the internet it is discovered that using DNS tunneling an user could avoid getting charged for the internet use. This essay will first explain some basics of how DNS work and the important elements that is exploited and the use of DNS tunneling, specifically the use on mobile networks.





\section{Problem Discussion}
\subsection{DNS}
There are many different record types in the DNS, most commonly is \texttt{'A'} and \texttt{'AAAA'} records which contains the ipv4 and the ipv6 addresses respectively. You also have a \texttt{'CNAME'} record which translate a domain name into another which is associated with an IP-address, e.g. if you enter \texttt{www.aftenposten.no} in the browser to visit the website of Aftenposten, the first response of the DNS server is a CNAME which refers you to the domain \texttt{aftenposten.no}. The network then send a new request to the DNS with aftenposten.no and receives an 'A' record with the IP-address. Now DNS has over 30 different record types, each has a different purpose and different size of payloads which is an important feature when it comes to DNS tunneling.

DNS has a hierarchical build. Each server sends the request to the next server with more specific information about the domain name you want to reach, going in a postfix order. 



\subsection{DNS Tunneling}





\subsection{Requirements of Form}

\begin{wrapfigure}[10]{r}{0.25\textwidth}
  \centering
  \includegraphics[scale=0.2]{Alice} %Note: no use of .jpg file ending
  \vspace{-0.2cm}
  \caption{A ``wrapped'' figure with the text.}
  \label{fig:Alice}
\end{wrapfigure}

We set the following requirements with respect to format:
\begin{enumerate}
	\item This \LaTeX\ template must be used.
	\item The whole document must be limited to~3~A4~pages of text (references not included).  A technical essay different from three -3- pages of text will be returned to the author
to be cut or enhanced to three pages.
	\item One or more illustrations in the form of figures, tables or diagrams must be included (see \hyperref[fig:Alice]{Figure~\ref*{fig:Alice}}).
	\item The entries in your \hyperref[sec:references]{reference} should be structured as follows:\\
	Author/Origin. \textit{Title}. Where, and when published.
	\item The submitted file format must be pdf.
\end{enumerate}


\paragraph{Structure}  If you want  further substructure than sections and subsections, then you can use
a paragraph title like here.  This will look much better than introducing another numbered level of subsubsection.

\subsection{Requirements of Content}
 The text should be intelligible, logical, interesting and easy to read. Write for
your fellow engineering students, and assume that the reader has the same general theoretical
background as yourself. Use definitions, facts and logical argumentation. 

Interpret and refine the title, discuss the problem and intended scope in the
introduction. In the text, do not introduce facts that are not analysed later and
that are not relevant to your problem. Bring your \emph{own analysis and thinking} into
the essay. If you can, bring in new ideas. When you include tables or figures (see \hyperref[fig:linksys]{Figure~\ref*{fig:linksys}}), 
they should be referred to and explained in the text.

While it is allowed to cite Wikipedia or another web page~\cite{Wikipedia:citation,Daborn}, 
it is not recommended.
It is much better if you can refer to a technical article or paper~\cite{Kortvedt:2009}.

\section{Conclusion}
The submitted technical essays will be graded and contribute 20\% to the final grade of the course. The three best essays will be honoured with publication on ItsLearning, edited if necessary, and become part of this year's syllabus.

\begin{figure}[hbp]
	\centering
	\begin{subfigure}[b]{0.3\textwidth}
		\centering
		\includegraphics[scale=0.20]{linksys}
		\caption{Scale=0.20}
	\end{subfigure}
	\begin{subfigure}[b]{0.3\textwidth}
		\centering		
		\includegraphics[scale=0.15]{linksys}
		\caption{Scale=0.15}	
	\end{subfigure}
	\begin{subfigure}[b]{0.3\textwidth}
		\centering		
		\includegraphics[scale=0.10]{linksys}
		\caption{Scale=0.10}
	\end{subfigure}
	\caption{The Linksys WRT54G line of routers include both  802.3 Ethernet and 802.11b/g wireless LAN.}
	\label{fig:linksys}
\end{figure}

% It is highly recommended to keep your references in a separate file (here called "references.bib")
% However if you want to create the reference section manually, 
% uncomment the lines below starting at "\begin{thebibliography} ... etc"
\bibliographystyle{plain}
\bibliography{references}\label{sec:references}



%\begin{thebibliography}{N}\label{sec:references}
%\bibitem{wiki1} Wikipedia. \textit{Citation.} Available at \url{http://en.wikipedia.org/wiki/Citation}.
%
%\bibitem{Daborn} Gordon Baxter, Jon Lewis and Ishbel Duncan. \textit{What is a Technical Essay?}
%Available online at \url{http://ishbel.host.cs.st-andrews.ac.uk/WhatisaTechnicalEssay.pdf}.
%
%\bibitem{Kortvedt} Henning Kortvedt and Stig Frode Mj{\o}lsnes. \textit{Eavesdropping Near Field Communication}.  
%In The Norwegian Information Security Conference (NISK 2009) Proceedings, pp. 57-68.  Tapir Akademiske Forlag, 2009.
%
%\end{thebibliography}  


\end{document} 
